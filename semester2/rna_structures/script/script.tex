\documentclass[12pt]{article}
\usepackage[utf8]{inputenc}
\usepackage{color}
\usepackage{hyperref}
\hypersetup{
    colorlinks=true,
    linktoc=all,
    linkcolor=black,
}
%Gummi|065|=)
\title{\Huge\textbf{Bioinformatik von RNA- und Proteinstrukturen}}
\author{}
\date{}

% set title of table of contents
\renewcommand*\contentsname{Inhalt}

% https://www.sharelatex.com/learn
% http://web.ift.uib.no/Teori/KURS/WRK/TeX/symALL.html

\begin{document}

\begin{titlepage}

\maketitle
\thispagestyle{empty}
\end{titlepage}
\newpage

\begin{titlepage}
\tableofcontents
\thispagestyle{empty}
\end{titlepage}
\newpage

\section{Einleitung}

Struktur: Form $\rightarrow$ Funktion\\
Funktion folgt Form, Form folgt Sequenz\\
Proteine, RNA, DNA: Sequenzen\\
\\
\underline{4 Strukturlevels:}
\begin{itemize}
	\item primäre Struktur (Sequenz): 1 Dimension
	\item sekundäre Struktur (grobe Annährung an Struktur): 2 Dimensionen
	\item tertiäre Struktur (räumliche Strukur): 3 Dimensionen
	\item quartiäre Struktur (räumliche Anordnung von interagierenden Strukturen): 4 Dimensionen
\end{itemize}
%
Behandlung hauptsächlich 2D

\section{Strukturvorhersage}

\subsection{Nussinov}

\subsection{Turner-Modell (Nearest-Neighbor-Modell)}

\subsection{Zuker-Algorithmus}

\subsubsection{suboptimales Falten}

\subsection{Wuchty-Algorithmus}

\subsubsection{Wuchty-Backtracking}

\subsection{McCaskill}

\subsection{stochastisches Backtracking}

\subsection{Strukturvorhersagen verbessern}

\subsection{Konsensusstrukturvorhersagen}

\subsection{Wie kann RNA evolvieren?}

\subsubsection{Neutrale Netzwerke}

\subsubsection{SHAPE-Abstraktion}

\subsubsection{Energielandschaften}

\subsubsection{Faltungskinetik}

\subsubsection{Barriers Trees}

\subsubsection{Flooding-Algorithmus}

\subsubsection{Co-transcriptional folding}

\end{document}