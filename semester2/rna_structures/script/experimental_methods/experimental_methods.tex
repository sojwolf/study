\documentclass[12pt,a4paper]{article}
\usepackage[english,german]{babel}
\usepackage[utf8]{inputenc}
\usepackage{color}
\usepackage{hyperref}
\usepackage{mathtools}
\usepackage{amsmath}
\usepackage{graphicx}

\usepackage{geometry}
\geometry{
  left=3cm,
  right=3cm,
  top=3cm,
  bottom=4cm,
  bindingoffset=5mm
}

\setlength{\parindent}{0em} 
\hypersetup{
    colorlinks=true,
    linktoc=all,
    linkcolor=black,
    urlcolor=black
}

% Hurenkinder und Schusterjungenregel
\clubpenalty = 10000
\widowpenalty = 10000
\displaywidowpenalty = 10000

%Gummi|065|=)
\title{\Huge\textbf{experimentelle Methoden der Bioinformatik}}
\author{}
\date{}

% set title of table of contents
\renewcommand*\contentsname{Inhalt}

% https://www.sharelatex.com/learn
% http://www.math.ubc.ca/~cautis/tools/latexmath.html
% http://www.golatex.de/wiki/Kategorie:Befehlsreferenz
% https://en.wikibooks.org/wiki/LaTeX/Mathematics

\begin{document}

\begin{titlepage}

\maketitle
\thispagestyle{empty}
\end{titlepage}
\newpage

\begin{titlepage}
\tableofcontents
\thispagestyle{empty}
\end{titlepage}
\newpage

%\section{Motifsuche}
%\subsection{ChIP-chip und ChIP-seq}
%\subsection{crosslinking}
%\subsection{sonication}
%\subsection{Selektion mittels Antikörpern}
%\subsubsection{Eigenschaften von Antikörpern}
%\subsection{reverse crosslinking}
%\subsection{Chiphybridisierung}
%\subsection{Sequencing}
\section{Vorlesung 20.04.2016}

\subsection{Mechanorezeptoren von C. elegans}

\subsection{Neurogenese in der Entwicklung von Drosophila}

\newpage

\section{Vorlesung 20.04.2016}

\subsection{Mechanorezeptoren von C. elegans}

\subsection{Neurogenese in der Entwicklung von Drosophila}

\newpage

\section{Peak Calling}

\section{CLIP-Seq}

\subsection{ICLIP}

\section{PAR-CLIP}

\section{Protein-Protein-Interaktion}

\section{Tandem Affinity Purification (TAP)}

\subsection{Local clique merging algorithm (LCMA)}

\subsection{Clique Finding Algorithzm (CFA)}

\section{RNA structure probing}

\subsection{chemical probing}

\section{Vorlesung 20.04.2016}

\subsection{Mechanorezeptoren von C. elegans}

\subsection{Neurogenese in der Entwicklung von Drosophila}

\newpage

\section{Vorlesung 20.04.2016}

\subsection{Mechanorezeptoren von C. elegans}

\subsection{Neurogenese in der Entwicklung von Drosophila}

\newpage

\section{Vorlesung 20.04.2016}

\subsection{Mechanorezeptoren von C. elegans}

\subsection{Neurogenese in der Entwicklung von Drosophila}

\newpage

\section{Vorlesung 20.04.2016}

\subsection{Mechanorezeptoren von C. elegans}

\subsection{Neurogenese in der Entwicklung von Drosophila}

\newpage

\section{Vorlesung 20.04.2016}

\subsection{Mechanorezeptoren von C. elegans}

\subsection{Neurogenese in der Entwicklung von Drosophila}

\end{document}