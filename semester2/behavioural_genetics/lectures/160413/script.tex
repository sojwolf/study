\section{Vorlesung 13.04.2016}

\subsection{Prinzipien der Verschaltung}
\begin{itemize}
	\item Konvergenz redundanter Einganssignale (siehe Bilder Vorlesung)
		\begin{itemize}
			\item Kein Input- kein Output
			\item Idealer Input liefert idealen Output
			\item Realistischer Input liefert verzerrten Output
	\end{itemize}
\end{itemize}

Divergenz/ Konvergenz verschiedener Eingangssignale (siehe Bilder Vorlesung)
\begin{itemize}
	\item Erlaubt Detektion von Kombinationen
	\item Erhöht die Anzahl der Dimensionen
	\item Erhöht das Auflösungsvermögen
\end{itemize}

Prinzipien der Verschaltung (siehe Bilder Vorlesung)
\begin{itemize}
	\item Negative Rück-Koppelung: erzeugt Schwingung
	\item Negative Vor-Koppelung: nimmt erste zeitliche Ableitung
	\item Laterale Hemmung: verstärkt Unterschiede (Wo ich bin, da kannst du nicht sein!)
	\item Efferenzkopie: unterscheidet Selbst von Nicht-Selbst
\end{itemize}

\subsection{Struktur von Nervensystemen}
Kenntnis der Struktur ermöglicht Untersuchung der Struktur - Funktions-Beziehung
\\\\
Wie?
\begin{itemize}
	\item Messung der Erregungsleitung, fMRI
	\item Untersuchung von Defekten, Ablation von Zellen / Strukturen
	\item Mutantenanalyse
	\item Genetische Manipulation
\end{itemize}

\textbf{Zentrale Frage der Entwicklungsneurogenetik:} Wie ist der Bauplan des NS in den Genen enthalten und wie sieht die Kontrolle dieser Gene aus?
\\\\
\textbf{Was ist überhaupt genetisch determiniert?} 

Antwort durch:
\begin{itemize}
	\item Vergleich genetisch identischer (isogener) Individuen
	\item Vergleich in bilateralsymetrischen Individuen
\end{itemize}

\textbf{Vorteil:} nicht nur genetische Identität, sondern auch weitgehend gleiche entwicklungsrelevante Umwelteinflüsse
\\\\
Vergleich parthenogenetischer Organismen
\\\\
\textbf{Parthenogenese:} ist eine Form der eingeschlechtlichen Fortpflanzung. Dabei entstehen die Nachkommen aus unbefruchteten Eizellen.